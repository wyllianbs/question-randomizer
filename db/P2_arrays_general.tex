\needspace{6\baselineskip} 
\item \rtask \ponto{\pt} Com relação aos tipos estruturados da linguagem de programação Python 3, avalie as afirmativas a seguir.

\begin{enumerate}[label={\Roman*.},leftmargin=*]
    \item A tentativa de utilizar conjunto (\lstinline[style=Python]|set|) como chave de um dicionário (\lstinline[style=Python]|dict|) retorna erro.
    \item O conjunto (\lstinline[style=Python]|set|) não permite elementos repetidos.
    \item O construtor do tipo \lstinline[style=Python]|list| retorna erro ao receber uma \textit{string} como argumento (entrada).
\end{enumerate}

Está correto o que se afirma em

\begin{answerlist}[label={\texttt{\Alph*}.},leftmargin=*]
    \ti I, apenas.
    \di II, apenas. % gabarito 
    \ti I e II, apenas.
    \ti II e III, apenas.
    \ti I, II e III.
\end{answerlist}


% Alternativa correta: B - II, apenas.
% 
% Vamos analisar cada uma das afirmativas para entender por que a alternativa B é a correta.
% 
% I. A tentativa de utilizar conjunto (set) como chave de um dicionário (dict) retorna erro.
% 
% Essa afirmativa está correta. Em Python, as chaves de um dicionário precisam ser de um tipo que seja hashable, ou seja, imutável e que possua um valor de hash constante durante seu tempo de vida. Conjuntos (sets) são mutáveis e, portanto, não podem ser usados como chaves de um dicionário. Se tentarmos, Python levantará um TypeError.
% 
% II. O conjunto (set) não permite elementos repetidos.
% 
% Essa afirmativa está correta. Em Python, conjuntos (sets) são coleções não ordenadas de elementos únicos. Isso significa que qualquer tentativa de adicionar elementos duplicados será ignorada, mantendo apenas um exemplar de cada elemento.
% 
% III. O construtor do tipo list retorna erro ao receber uma string como argumento (entrada).
% 
% Essa afirmativa está incorreta. Em Python, o construtor list() pode receber uma string como argumento sem problemas. Nesse caso, ele criará uma lista onde cada caractere da string será um elemento separado da lista. Por exemplo, list("abc") resultará em ['a', 'b', 'c'].
% 
% Portanto, as únicas afirmativas corretas são a II, o que torna a alternativa B - II, apenas a resposta correta.
