\needspace{12\baselineskip} 
\item \rtask \ponto{\pt} Analise o código Python a seguir.

\begin{lstlisting}[style=Python]
x = lambda a, b: a + [a[-1] + a[-2] + b]
y = [-1, 0]
for i in range(7):
	y = x(y, i)
print(y)
\end{lstlisting}

O resultado produzido pela execução desse código é:

\begin{answerlist}[label={\texttt{\Alph*}.},leftmargin=*]
    \ti \lstinline[style=Python]|[-1, 0, -1, -1, -2, -3, -5, -8, -13]|.
    \ti \lstinline[style=Python]|[0, -1, 4, 9, 18, 33, 51, 84]|.
    \ti \lstinline[style=Python]|[-1, 0, -1, -2, -3, -5, -8, -13, -21]|.
    \di \lstinline[style=Python]|[-1, 0, -1, 0, 1, 4, 9, 18, 33]|.
    \ti \lstinline[style=Python]|[-1, 0, -1, 4, 9, 18, 33, 51]|.
\end{answerlist}

% -> D
% [-1, 0, -1, 0, 1, 4, 9, 18, 33]


