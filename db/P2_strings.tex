\needspace{14\baselineskip} 
\item \rtask \ponto{\pt} Observe o código em Python a seguir.

\begin{lstlisting}[style=Python]
seq = 'AUUCCUUCTGG'
seq = seq.replace('A', 'G')
seq = seq.replace('U', 'T')
G = seq.count('G')
C = seq.count('C')
T = seq.count('T')
print(G, C, T)
\end{lstlisting}

Após a execução do código, o resultado impresso na tela será

\begin{answerlist}[label={\texttt{\Alph*}.},leftmargin=*]
    \ti \lstinline[style=Python]|G, C, T|.
    \ti \lstinline[style=Python]|3, 3, 5|.
    \ti \lstinline[style=Python]|G C T|.
    \di \lstinline[style=Python]|3 3 5|. % gabarito
    \ti \lstinline[style=Python]|5, 3, 3|.
\end{answerlist}
