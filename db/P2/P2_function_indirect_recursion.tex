\needspace{15\baselineskip} 
\item \rtask \ponto{\pt} Considere o código Python a seguir.

\begin{lstlisting}[style=Python]
def ABC(L, n):
    while True:
        if len(L) >= n:
            return L
        else:
            L.append(len(L) ** 2)

print(ABC([20], 10))
\end{lstlisting}

O resultado da execução desse código é:

\begin{answerlist}[label={\texttt{\Alph*}.},leftmargin=*]
    \ti \lstinline[style=Python]|[1, 4, 9, 16, 25, 36, 49, 64]|.
    \ti \lstinline[style=Python]|[1, 4, 9, 16, 25, 36, 49, 64, 81]|.
    \ti \lstinline[style=Python]|[20, 1, 4, 9, 16, 25, 36, 49, 64]|.
    \di \lstinline[style=Python]|[20, 1, 4, 9, 16, 25, 36, 49, 64, 81]|.
    \ti \lstinline[style=Python]|[20, 4, 9, 16, 25, 36, 49, 64, 81]|.
\end{answerlist}

