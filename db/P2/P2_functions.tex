\needspace{13\baselineskip} 
\item \rtask \ponto{\pt} No Python, funções são blocos de código identificados por um nome, que podem receber parâmetros predeterminados. Em relação às observações a serem consideradas na execução funções, está incorreta a seguinte afirmação:

\begin{answerlist}[label={\texttt{\Alph*}.},leftmargin=*]
    \ti os argumentos com padrão devem vir por último, depois dos argumentos sem padrão.
    \ti o valor do padrão para um parâmetro é calculado quando a função é definida.
    \di Os argumentos passados sem identificador são recebidos pela função na forma de um dicionário.
    \ti Aceitam \textit{Doc Strings}.
    \ti os parâmetros passados com identificador na chamada da função devem vir no fim da lista de parâmetros. 
\end{answerlist}
