\needspace{26\baselineskip} 
\item \rtask \ponto{\pt} Considere o seguinte código em Python:

\begin{lstlisting}[style=Python]
M = {
    "A": "B",
    "C": True,
    "D": 582,
    "E": ["A", "E", "I", "O", "U"],
    "F": {
        "A": "B",
        "E": ["R", "T", "S", "M", "L"],
        "D": 386
    }
}

def Fun(var1=M):
    var2 = var1["D"]
    return var1["E"][2] + var1["A"] + f"{var2}"

p = Fun(M["F"])
q = Fun()
print(p + "-" + q)
\end{lstlisting}

Qual é a saída correta desse código?

\begin{answerlist}[label={\texttt{\Alph*}.},leftmargin=*]
    \ti \lstinline[style=Python]|IB386-SB582|.
    \ti \lstinline[style=Python]|TB386-IB582|.
    \di \lstinline[style=Python]|SB386-IB582|. % gabarito
    \ti \lstinline[style=Python]|SB386-SB386|.
    \ti \lstinline[style=Python]|IB582-SB386|.
\end{answerlist}

