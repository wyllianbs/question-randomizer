\needspace{6\baselineskip} 
\item \rtask \ponto{\pt} Considere-se o código a seguir.

\begin{lstlisting}[style=Python]
numeros = (1,5,7,35,50,75,80,90)
print(numeros[4])
\end{lstlisting}

Ao se executar o código apresentado, o resultado será a impressão em tela do valor 50.

% V
{\setlength{\columnsep}{0pt}\renewcommand{\columnseprule}{0pt}
\begin{multicols}{2}
\begin{answerlist}[label={\texttt{\Alph*}.},leftmargin=*]
    \ifnum\gabarito=1\doneitem[V.]\else\ti[V.]\fi % gabarito
    \ti[F.]
\end{answerlist}
\end{multicols}
}

% JUSTIFICATIVA - Certo. As tuplas em Python iniciam na posição zero, então dessa forma a quinta posição da esquerda para direita será o número 50

