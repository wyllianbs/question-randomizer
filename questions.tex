\needspace{6\baselineskip} 
\item \rtask \ponto{\pt} Na linguagem Python, são consideradas sequências mutáveis as

\begin{answerlist}[label={\texttt{\Alph*}.},leftmargin=*]
    \ti tuplas.
    \ti cadeias.
    \ti \textit{strings}.
    \di listas. % gabarito
    \ti \textit{ranges}.
\end{answerlist}


\needspace{6\baselineskip} 
\item \rtask \ponto{\pt} Analise o código Python a seguir.

\begin{lstlisting}[style=Python]
x = [1,2,3,4,5]
print (x[::-1])
\end{lstlisting}

Assinale a opção que indica a saída produzida pela execução desse código.

\begin{answerlist}[label={\texttt{\Alph*}.},leftmargin=*]
    \ti \lstinline[style=Python]|1|.
    \ti \lstinline[style=Python]|5|.
    \di \lstinline[style=Python]|[5,4,3,2,1]|. % gabarito
    \ti \lstinline[style=Python]|[5,1]|.
    \ti \lstinline[style=Python]|[1,2,3,4,5]|.
\end{answerlist}


\needspace{26\baselineskip} 
\item \rtask \ponto{\pt} Considere o seguinte código em Python:

\begin{lstlisting}[style=Python]
M = {
    "A": "B",
    "C": True,
    "D": 582,
    "E": ["A", "E", "I", "O", "U"],
    "F": {
        "A": "B",
        "E": ["R", "T", "S", "M", "L"],
        "D": 386
    }
}

def Fun(var1=M):
    var2 = var1["D"]
    return var1["E"][2] + var1["A"] + f"{var2}"

p = Fun(M["F"])
q = Fun()
print(p + "-" + q)
\end{lstlisting}

Qual é a saída correta desse código?

\begin{answerlist}[label={\texttt{\Alph*}.},leftmargin=*]
    \ti \lstinline[style=Python]|IB386-SB582|.
    \ti \lstinline[style=Python]|IB582-SB386|.
    \di \lstinline[style=Python]|SB386-IB582|. % gabarito
    \ti \lstinline[style=Python]|SB386-SB386|.
    \ti \lstinline[style=Python]|TB386-IB582|.
\end{answerlist}


\needspace{15\baselineskip} 
\item \rtask \ponto{\pt} Considere o código Python a seguir.

\begin{lstlisting}[style=Python]
def ABC(L, n):
    while True:
        if len(L) >= n:
            return L
        else:
            L.append(len(L) ** 2)

print(ABC([20], 10))
\end{lstlisting}

O resultado da execução desse código é:

\begin{answerlist}[label={\texttt{\Alph*}.},leftmargin=*]
    \ti \lstinline[style=Python]|[1, 4, 9, 16, 25, 36, 49, 64]|.
    \ti \lstinline[style=Python]|[20, 4, 9, 16, 25, 36, 49, 64, 81]|.
    \ti \lstinline[style=Python]|[1, 4, 9, 16, 25, 36, 49, 64, 81]|.
    \di \lstinline[style=Python]|[20, 1, 4, 9, 16, 25, 36, 49, 64, 81]|.
    \ti \lstinline[style=Python]|[20, 1, 4, 9, 16, 25, 36, 49, 64]|.
\end{answerlist}


\needspace{22\baselineskip} 
\item \rtask \ponto{\pt} A linguagem Python é de propósito geral, pois possui tipagem dinâmica e uma de suas principais características é permitir a fácil leitura do código fonte e exigir poucas linhas de código se comparado ao mesmo programa em outras linguagens. Considere o programa a seguir, que ilustra a criação e execução de um algoritmo desenvolvido em Python.

\begin{lstlisting}[style=Python]
a = 15
i = 1
soma = 0
resultado = 0
while i <= 6:
    resultado = a % i
    soma = soma + resultado
    i = i+1
print(soma)
\end{lstlisting}

Assinale a alternativa correta para o valor que será impresso ao final da execução do algoritmo.

\begin{answerlist}[label={\texttt{\Alph*}.},leftmargin=*]
    \di \lstinline[style=Python]|7|. % gabarito
    \ti \lstinline[style=Python]|5|.
    \ti \lstinline[style=Python]|6|.
    \ti \lstinline[style=Python]|12|.
    \ti \lstinline[style=Python]|4|.
\end{answerlist}


\needspace{13\baselineskip} 
\item \rtask \ponto{\pt} No Python, funções são blocos de código identificados por um nome, que podem receber parâmetros predeterminados. Em relação às observações a serem consideradas na execução funções, está incorreta a seguinte afirmação:

\begin{answerlist}[label={\texttt{\Alph*}.},leftmargin=*]
    \di Os argumentos passados sem identificador são recebidos pela função na forma de um dicionário.
    \ti os argumentos com padrão devem vir por último, depois dos argumentos sem padrão.
    \ti os parâmetros passados com identificador na chamada da função devem vir no fim da lista de parâmetros.
    \ti Aceitam \textit{Doc Strings}.
    \ti o valor do padrão para um parâmetro é calculado quando a função é definida.
\end{answerlist}


\needspace{8\baselineskip} 
\item \rtask \ponto{\pt} Assinale a opção abaixo que contém SOMENTE informações CORRETAS    .

\begin{answerlist}[label={\texttt{\Alph*}.},leftmargin=*]
    \di Dicionários em Python 3 preservam a ordem de inserção.
    \ti \lstinline[style=Python]|count(d)| retorna o número de elementos do \lstinline[style=Python]|dict| \lstinline[style=Python]|d|.
    \ti Utiliza-se \lstinline[style=Python]|array.add(x)| para adicionar \lstinline[style=Python]|x| a \textit{array}.
    \ti Python 3 possui retrocompatibilidade total com Python 2.
    \ti Python 3 não é compatível com cadeias de caracteres (\textit{strings}) Unicode.
\end{answerlist}


\needspace{16\baselineskip} 
\item \rtask \ponto{\pt} Considere o seguinte trecho de código na linguagem Python:

\begin{lstlisting}[style=Python]
def f(n):
    if n == 0:
        return 0
    if n == 1:
        return 1
    if n > 1:
        return f(n-1) * n
\end{lstlisting}

Ao chamar a função \lstinline[style=Python]|f|, passando-se como parâmetro o valor \lstinline[style=Python]|6|, o resultado esperado é:

\begin{answerlist}[label={\texttt{\Alph*}.},leftmargin=*]
    \ti Obtenção de um erro, uma vez que a função é chamada indefinidamente dentro dela mesma.
    \ti \lstinline[style=Python]|360|.
    \ti \lstinline[style=Python]|120|.
    \ti \lstinline[style=Python]|480|.
    \di \lstinline[style=Python]|720|.
\end{answerlist}


